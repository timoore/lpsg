Although performance optimization of OpenGL programs is a complex and
sometimes arcane subject, several strategies are always helpful:
reduce the number of calls to the OpenGL API; reduce the amount of
memory traffic from the CPU to the GPU; and minimize the number of
graphical state changes during rendering. lpsg is a library that helps
to implement these strategies.

OpenGL provides storage, called \emph{buffer objects}, that is
allocated on the GPU. Once initialized, the data in buffer objects
does not need to be copied from the CPU unless it is modifed. In
recent versions of OpenGL, buffer objects are the only mechanism for
supplying geometric data. Earlier versions do support rendering
directly from CPU memory, but buffer objects provide a performance
advantage in most cases. The usage of buffer objects is somewhat
complicated, with various options available for copying data into
buffer objects. lpsg provides an interface for allocating and
deallocating storage from buffer objects; it manages the actual
allocation of the buffer objects in OpenGL itself.

lpsg 
lpsg also provides support for the management of that graphical state which
is not uniquely associated with any one geometric object.

lpsg is not a scene graph!

